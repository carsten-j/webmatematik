%!TEX program = lualatex
\documentclass[11pt, a4paper]{article}

\usepackage{fontspec}
\usepackage{polyglossia}
\usepackage{hyperref}
%\usepackage{color}
\usepackage[svgnames]{xcolor}

\hypersetup{
     colorlinks   = true,
     citecolor    = gray,
     urlcolor=Blue
}

\begin{document}

\section{Introduktion til multipel regression}

Fra simpel \href{http://www.webmatematik.dk/lektioner/matematik-b/regression}{lineær
regressions} analyse ved vi, hvordan man med
\href{http://www.webmatematik.dk/lektioner/matematik-b/regression}{mindste kvadraters metode}
bestemmer den lineære funktion, som passer bedst til en rækker punkterne i 2D planen.

Vi har altså en række punkter \((y_i, x_i)\) for \(i=1,\ldots,n\) og ønsker at bestemme
\(a\) og \(b\) på en sådan måde, at den lineære funktion
\begin{displaymath}
  y = b + a x
\end{displaymath}
ligger så tæt på alle punkterne \((y_i, x_i)\) som muligt.

Verden er dog sjældent så simpelt indrettet, at man kan beskrive en afhængig variabel \(y\)
med kun en enkelt forklarende variabel \(x\).

Multipel regression er en udvidelse af ovenstående, hvor vi i stedet for en enkelt forklarende
variabel, har to eller flere forklarende variable.

eksempel

For en række punkter (observationer) \(y_i,x_{1,i},x_{2,i},\ldots,x_{p,i}\) hvor \(i=1,\ldots,n\) øsnker vi
at bestemme konstanter \(b_0,b_1,b_2,\ldots,b_p\), så funktionen
\begin{displaymath}
  y = b_0 + b_1 x_1 + b_2 x_2 + \dots b_p x_p
\end{displaymath}
ligger så tæt på alle punkterne \(y_i,x_{1,i},x_{2,i},\ldots,x_{p,i}\) som muligt.


mere formel definition

sammenlign med lineær regression

\subsubsection{Historie}


\subsection{Korrelationskoefficient}

\subsection{Determinationskoefficient}

\subsection{Residualplot}


\subsection{Konfidensinterval for parametre}


\subsection{CAS fx R}

\subsection{brug din fornuft}
counterexample

kig på graf

\subsection{model selection}

\subsection{adjusted \(R^2\)}


\end{document}