\documentclass{article}
\usepackage[utf8]{inputenc}

\title{Statistik og sandsynlighedsregning}


\begin{document}

\maketitle

\section{Konfidensintervaller}
Hvis man ønsker at beregne den gennemsnitlige højde for alle studerende ved gymnasielle uddannelser i Danmark, kan vi fx udvælge en 1.g. gymnasieklasse i Københavner området og beregne middelværdien for de studerendes højde i denne klasse. Den beregnede middelværdi er så et bud på den gennemsnitlige højde for alle studerende ved gymnasielle uddannelser. Men hvis vi nu vælger en anden klasse og beregner middelværdien igen, så får vi sikkert en værdi som ligger tæt på den første middelværdi uden at den er identisk med denne. Middelværdien afhænger altså af den population (i dette tilfælde en bestemt klasse), som vi bruger til at udregne middelværdien. Uanset hvordan vi vælger vores population vil den beregnede middelværdi aldrig være lig med den præcise gennemsnitshøjde.
\subsection{Formel}

\subsection{Eksempel}

\end{document}
