
med ukendt middelværdi
\(\mu\) og spredning på \(\)

Lad os bruge betegnelsen \(x_1, x_2, \ldots, x_n\) om højden af de \(n\) elever i
den ene stikprøve. Middelværdien er her
\begin{equation}
  \bar{x} = \sum_{i=1}^n x_i
\end{equation}

Variansen
\begin{equation}
  \Var(x) = \frac{1}{n} \sum_{i=1}^n (x_i - \bar{x})^2
\end{equation}
fortæller os noget om, hvor lang eller kort observationerne ligger fra
middelværdien. Spredningen er lig med kvadratroden af variansen, altså
\begin{equation}
  \sigma(x) = \sqrt{\Var(x)}
\end{equation}

Usikkerheden i middelværdien

fordeling af middelværdien

hvad er middelværdien og spredning af middelværdier?


som ligger tæt på den første
middelværdi uden at den er identisk med denne. Middelværdien afhænger altså af den population (i dette tilfælde en bestemt klasse), som vi bruger til at udregne middelværdien. Uanset hvordan vi vælger vores population vil den beregnede middelværdi aldrig være lig med den præcise gennemsnitshøjde.
\subsection{Formel}

\subsection{Eksempel}



punkt estimat

SE
jo større stikprøve des mindre SE


fortolkning

ahss
5.2.5



formel


Hvor kommer de 1.96 fra?


sandsynlighed og inferens



\subsection{SEM}
I formlen for konfidensintervallet omtales størrelsen
\begin{equation}
\frac{\SD}{\sqrt{n}}
\end{equation}
tit som ``Standard Error of Mean'' (og forkortes \(\SEM\)). Det kan oversættes til dansk
som noget i stil med enten ``usikkerhed på middelværdien'' eller ``standardfejl''.
Dette hænger sammen med vores betragtning ovenfor, hvor vi bemærkede at to
forskellige stikprøver gav to forskellige estimater for middelværdien. Dermed har
vi en vis usikkerhed på estimatet.

Stikprøvens størrelse kan have indflydelse på \(\SEM\). Læg mærke til at i formlen for
\(\SEM\) divideres med kvadratroden af stikprøvens størrelse \(n\). Ofte vil en
større stikprøve derfor give en mindre ``standardfejl''.




